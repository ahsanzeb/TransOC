\documentclass[a4paper]{revtex4-1} %
\usepackage{amsmath}
\usepackage{graphicx}
\usepackage{braket}
\usepackage{comment}
\usepackage{hyperref}
\usepackage{soul,color}
\usepackage{cleveref}
%\usepackage{stfloats} % table on both columns
\usepackage{multirow}

\newcommand{\da}{\downarrow}
\newcommand{\ua}{\uparrow}
\newcommand{\ad}{\hat{a}^\dagger}
\newcommand{\an}{\hat{a}^{}}
\newcommand{\bd}{\hat{b}^\dagger}
\newcommand{\bn}{\hat{b}^{}}
\newcommand{\sd}{\hat{\sigma}^{+}}
\newcommand{\sn}{\hat{\sigma}^{-}}
\newcommand{\mbf}[1]{\mathbf{#1}}
\newcommand{\del}{\hat{\Delta}^{}}

\newcommand{\ind}[1]{\mathcal{A}_{#1}}



\newcommand{\dm}{\widetilde{{\Delta}^{}}}
\newcommand{\A}{\mathbf{A}}
\newcommand{\B}{\mathbf{B}}

\newcommand{\EPF}{\ket{\text{EP-F}}}
\newcommand{\EP}{\ket{\text{EP}}}
\newcommand{\PP}{\ket{\text{PP}}}
\newcommand{\tPS}{\ket{\text{2P}}}
\newcommand{\tPF}{\ket{\text{2P-F}}}

\newcommand{\az}[1]{{\color{magenta}{#1}}}
\newcommand{\jk}[1]{{\color{red}{#1}}}
\newcommand{\pk}[1]{{\color{blue}{#1}}}
\newcommand{\q}[1]{{\color{cyan}{#1}}}

\newcommand{\symset}{\mathcal{S}}
\newcommand{\cntset}{\mathcal{P}} 
\newcommand{\indset}{\mathcal{I}} 

\newcommand{\maptoN}{{{\mathcal{M}}}_{N-1 \mapsto N}}
\newcommand{\maptoNa}{{\mathcal{M}}_{N-2 \mapsto N-1}}

\newcommand{\kb}{\ket{0}\!\bra{0}}
\newcommand{\klb}{\ket{\lambda}\!\bra{0}}
\newcommand{\kbl}{\ket{0}\!\bra{\lambda}}
\newcommand{\klbl}{\ket{\lambda}\!\bra{\lambda}}

\graphicspath{{./figs/}}
\usepackage{float}

\begin{document}

\title{Supplimentary Materials: Charge transport in organic microcavities: phonon-assisted zener tunnelling under strong light-matter coupling}

\author{M. Ahsan Zeb}
\affiliation{SUPA, School of Physics and Astronomy, University of St Andrews, St Andrews, KY16 9SS, United Kingdom}
\author{Peter G. Kirton}
\affiliation{SUPA, School of Physics and Astronomy, University of St Andrews, St Andrews, KY16 9SS, United Kingdom}
\author{Jonathan Keeling}
\affiliation{SUPA, School of Physics and Astronomy, University of St Andrews, St Andrews, KY16 9SS, United Kingdom}
\date{\today}

\begin{abstract}
\end{abstract}

%\maketitle
\section{Overview}

\subsection{Nomenclature}
\begin{itemize}
\item D: doubly occupied site
\item $\phi$: empty/unoccupied site
\item Active: singly occupied site, $\da$ or $\ua$, that couples to the cavity.
\item Channels: H-H, L-L, L-H, H-L channels for electron jumps.
\item Bulk hopping parameters: $t_h$, $t_l$, $t_{lh}$, $t_{hl}$ for H-H, L-L, L-H, H-L channels.
\item Contact hopping parameters: $J_{h,R}$, $J_{l,R}$, $J_{h,L}$, $J_{l,L}$ for RC-H, RC-L, LC-H, LC-L channels, where RC and LC stand for right and left contats.
\item Sites: molecules
\item classical hop: electron hop between molecules or contact and molecule, treated classically, i.e., no interference effects between different hops.
\item quantum transition: transition between eigenstates of the quantum mechanically described system, i.e., active sites coupled to the cavity.
\item Degenerate sectors: degenerate sectors in the energy eigenvalues of coupled light-matter system. 
\item Transition matrix: transition matrix due to a classical hopping process 
that gives the bare amplitudes for the quantum transitions for given initial and final states.
\item Penalty function: a penalty function suppressing the transition rate by a Boltzmann factor if there is a positive energy cost for the transition. This is widely used in Monte Carlo simulations to allow, in a statistical way, the energy exchange between the system and the environment. 
\end{itemize}


\subsection{Outline}

At every iteration, steps:\\
\begin{enumerate}
\item 
{\bf The sets of allowed sites for all kind of processes:}
From the overall state of the system
that gives
the sets of D, $\phi$ and active sites, 
the sets of sites or pairs of sites for various hopping processes
is calculated. Electron hops to the right and left are treated separately.
For example, suppose two neighbouring sites are found in D and $\phi$ sets
with $\phi$ on the right of the D in the lattice, 
then they are added to the list for (D,$\phi$) pair annihilation with electron hopping to the right.


\item
{\bf The transition matrices are calculated for all available sites and channels for allowed processes.}
For the quantum mechanically described active sites that are coupled to the cavity,
their order in the basis set does not have to be the same as in the lattice (position space)
and is taken into account when computing transition matrices for various processes.

\item
{\bf  The transition amplitudes are computed that are used with corresponding energetic penalties 
to find the transition rates.}
The energetic penalties consider all relevant changes in the energy, possible contributions are:
the difference between energies of initial and final quantum
and classical states (D has an energy $\omega_0$),
energetic barriers at the contacts, energy gain/lost due to the applied electric field,
and finally the vibrational energy of an active site for hops from it if vibrational assistance is included.

\item 
{\bf  A transition is selected stochastically and the quantum state and the sets of D, $\phi$ and $\mathcal{A}$ is changed accordingly.}
After the transition, it is assumed that the interaction with the environment, low energy vibrations,
quickly relaxes the system to its lower polariton state.
So the state is changed accordingly and the loop restarts again from point $1$ above.

\end{enumerate}



\section{Transition Matrices}
%\subsection{Definitions}
%$\mathcal{A}_{n}$ Index in the list of active sites.
%$\mathcal{L}_{n}$ Index in the list of lattice sites.
%Spin sectors:\\
For $N$ active sites and $m$ excitations,
the basis states are made of sets of basis with $k\in [0,min(m,N)]$ sites in $\ua$ state
and indexed in lexicographic order.
The transition matrix maps are calculated for each of such set and combined to make full matrix.


\subsection{D hops right/left}
An electron hops from a D site to an active site.
The active site becomes D and vice versa, the index of the initially active site is assigned to the new active site. 
\\Transition matrix elements:\\
The basis states that have final site $\ua$ would allow 
an electron hopping to its HOMO level and create a D.
This hop leaves the initially D site in $\ua$ (channel 1) or $\da$  (channel 3).
Similarly,
the basis states that have final site $\da$ would allow 
an electron hopping to its LUMO level and create a D.
This hop leaves the initially D site in $\da$ (channel 2) or $\ua$  (channel 4).



\subsection{$\phi$ hops right/left}
An electron hops from an active site to a $\phi$.
The active site becomes $\phi$ and vice versa, the index of the initially active site is assigned to the new active site. 
\\Transition matrix elements:\\
The basis states that have the active site $\ua$ would allow 
the electron hopping from it and change it to $\phi$.
This hop makes the initially $\phi$ site in $\ua$ (channel 2) or $\da$  (channel 3).
Similarly,
The basis states that have the active site $\da$ would allow 
the electron hopping from it and change it to $\phi$.
This hop makes the initially $\phi$ site $\da$ (channel 1) or $\ua$  (channel 4).

\subsection{(D,$\phi$) creation: }
For two adjacent active sites,
we can have four configurations:
$(\ua,\da)$, $(\da,\ua)$,$(\ua,\ua)$,$(\da,\da)$ with left and right sites in (left,right) order.
These will make (D,$\phi$) pair with D on the left 
with the hopping of an electron from right to left 
via H-H, L-L, L-H, and H-L channels.
Similarly,
a (D,$\phi$) pair with D on the right is created from these configurations
with the hopping of an electron from left to right
via L-L, H-H, L-H, and H-L channels, respectively.
\\Transition matrix elements:\\
The map between basis states of initial and final Hilbert spaces
$\mathcal{H}_{N,m}$ and $\mathcal{H}_{N-2,m^\prime}$, 
where $m^\prime=m,m,m-1,m-2$ for H-H, L-L, L-H, H-L channels,
can be calculated by starting from the final basis states in $\mathcal{H}_{N-2,m^\prime}$ 
and inserting the two active sites
at their position in desired configuration ($(\ua,\da)$, etc.) to make the initial basis state.
Let's call the combination of $k$ ($k\in [0,min(m,N)]$) $\ua$ sites in the latter $set$ and its lexicographic index
$x$ in this $k-\ua$-sector, then
\begin{eqnarray*}
x = c^N_ k - 
   \sum_{p=0}^{k-1}c^{N - set(p)}_{ k - p},
   %   c^n_r &=& \frac{n!}{r!(n-r)!}.
\end{eqnarray*}  
 where, $c^n_r = {n!}/{r!(n-r)!}$.
  This index is then shifted up by an amount 
$\sum_{k^\prime=0}^{k-1} c^N_{k^\prime}$ to obtain the absolute index of the considered basis state.









\subsection{(D,$\phi$) annihilation: }
This is just the reverse of (D,$\phi$) creation described above where two active sites are created
from a pair of D and $\phi$. The freshly added active sites are given the indices $N+1,N+2$
and the map for the transition matrix is computed just like (D,$\phi$) creation case.
Since the electron hopping does not occur on active sites in this case,
a single (D,$\phi$) pair can be used to calculate the transition matrix and amplitudes.


\subsection{ D/$\phi$ creation at contacts: }
If the site adjacent to a contact is active, 
it can gain or loose an electron via hopping from/to the contact to become a D or $\phi$.
The basis states with this site $\da$ 
can get an electron from the contact (hopping parameter $J_{l,R/L}$; $R and L$ for right and left contacts)
and become D or it can loose the electron to the contact (hopping parameter $J_{h,R/L}$)
and become $\phi$
resulting in one less active site $N\rightarrow N-1$ and $m$ unchanged.
That is, the final state is in $\mathcal{H}_{N-1,m}$. 
The transition matrix map is obtained simply by starting with the basis in $\mathcal{H}_{N-1,m}$,
inserting an $\da$ in correct position to make the state in $\mathcal{H}_{N,m}$,
and calculating its lexicographic index.
Similarly, the basis states with this active site $\ua$ 
are linked via $J_{h,R/L}$ and $J_{l,R/L}$ to 
the basis in $\mathcal{H}_{N-1,m}$
for creation of D and $\phi$ respectively.
The transition matrix map is calculated the same way.


\subsection{ D/$\phi$ annihilation at contacts: }
These are just the reverse of D/$\phi$ creation at the contacts.
A D or $\phi$ site adjacent to a contact becomes active on loosing or gaining an electron to/from it.
Starting with D, 
the basis states in $\mathcal{H}_{N,m}$
are linked via $J_{l,R/L}$ ($J_{h,R/L}$) to 
the basis in $\mathcal{H}_{N+1,m^\prime}$ with $m^\prime=m\,(m+1)$
with this site being $\da$ ($\ua$).
For a $\phi$, 
the same transition map results with $J_{l,R/L} \leftrightarrow J_{h,R/L} $.




\bibliography{organic}

\end{document}

\begin{acknowledgments}
  JK and MAZ acknowledges financial support from EPSRC program ``Hybrid
  Polaritonics'' (EP/M025330/1).  PGK acknowledges support from EPSRC
  (EP/M010910/1). 
\end{acknowledgments}

